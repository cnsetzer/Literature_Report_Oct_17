\documentclass[11pt]{article}
\usepackage[utf8]{inputenc}
\usepackage{xcolor}
\usepackage{soul,color}
\DeclareRobustCommand{\hlcyan}[1]{{\sethlcolor{cyan}\hl{#1}}}
\usepackage[margin=18.5mm,bottom=23mm,top=21mm,paperwidth=210mm,paperheight=297mm,headsep=0.75cm]{geometry}
\usepackage{fancyhdr}
\pagestyle{fancy}
\lhead{\today}
\rhead{Christian N. Setzer} 
\renewcommand\headrule{\vspace{3pt}\hrule height 1pt width\headwidth \vspace{-10pt}}
\title{Research Proposal: SU Obs Cos}
\begin{document}
\hspace{-6mm}{\Large\bf{Literature Report}}
\par
\vspace{7pt}
Transient science with the Large Synoptic Survey Telescope (LSST) will be highly dependent on the ability to characterize the target populations and the cadence with which the survey area is sampled. These are restricted by what we know about the physics of these transient systems, the competition with other science cases for specific observing strategies and the physical limitations from observing conditions and instrumentation. 
\section{General Cadence Considerations}

\section{Physical Limitations}


\subsection{Supernovae}

\subsection{Kilonovae}

\section{Addressing the Detection of Kilonovae with LSST}

\subsection{Photometric Classification}

\section{Areas for Future Work}
Better understanding of what a representative sample for Kilonovae. 
Understand the desired sampling for photometric classification using SNmachine. Which parts of the light curve are sampled. Does this affect SN or KN classification?
Lightcurves are quite featureless. Would color curves help? Colors do evolve faster than individual lightcurves, but show what seem to be more "features". 

\end{document}

