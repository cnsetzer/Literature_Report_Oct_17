\documentclass[11pt]{article}
\usepackage[utf8]{inputenc}
\usepackage{xcolor}
\usepackage{soul,color}
\DeclareRobustCommand{\hlcyan}[1]{{\sethlcolor{cyan}\hl{#1}}}
\usepackage[margin=18.5mm,bottom=23mm,top=21mm,paperwidth=210mm,paperheight=297mm,headsep=0.75cm]{geometry}
\usepackage{fancyhdr}
\pagestyle{fancy}
\lhead{\today}
\rhead{Christian N. Setzer} 
\renewcommand\headrule{\vspace{3pt}\hrule height 1pt width\headwidth \vspace{-10pt}}
\title{Research Proposal: SU Obs Cos}
\begin{document}
\hspace{-6mm}{\Large\bf{Literature Report}}
\par
\vspace{7pt}
Transient science with the Large Synoptic Survey Telescope (LSST) will be highly dependent on the ability to characterize the target populations and the cadence with which the survey area is sampled. These are restricted by what we know about the physics of these transient systems, the competition with other science cases for specific observing strategies and the physical limitations from observing conditions and instrumentation. 
\section{General Cadence Considerations} %1.5 pages
The observational strategy or cadence that will be adopted for the LSST survey is still a work in progress. However, there are some main features which will be consistent throughout the optimization of cadence strategies. These regard primarily the Wide Fast Deep (WFD) survey strategy. What this actually entails are the generic features of the cadence which make this wide, fast, and deep. These descriptions would seem to something that any telescope would want the capability of having but what it signifies is that we are now in an area of Astronomy where we can simultaneously acheive all these goals. Covering a large sky area at a pace that would allow for the detection of changes in the sky on short timescales out to a distance that is beyond our local galactic environment make this a WFD survey. \par



\section{Physical Constraints} % 3 page


\subsection{Supernovae}

\subsection{Kilonovae}

\section{Addressing the Detection of Kilonovae with LSST} % 2 page

\subsection{Photometric Classification}

\section{Areas for Future Work} % 1 page
Better understanding of what a representative sample for Kilonovae. 
Understand the desired sampling for photometric classification using SNmachine. Which parts of the light curve are sampled. Does this affect SN or KN classification?
Lightcurves are quite featureless. Would color curves help? Colors do evolve faster than individual lightcurves, but show what seem to be more "features". 

\end{document}

