\documentclass[11pt]{article}
\usepackage[utf8]{inputenc}
\usepackage{xcolor}
\usepackage{soul,color}
\usepackage{hyperref}
\usepackage{natbib}
\DeclareRobustCommand{\hlcyan}[1]{{\sethlcolor{cyan}\hl{#1}}}
\usepackage[margin=18.5mm,bottom=23mm,top=21mm,paperwidth=210mm,paperheight=297mm,headsep=0.75cm]{geometry}
\usepackage{fancyhdr}
\pagestyle{fancy}
\lhead{\today}
\rhead{Christian N. Setzer} 
\renewcommand\headrule{\vspace{3pt}\hrule height 1pt width\headwidth \vspace{-10pt}}
\title{Initial Literature Report}
\begin{document}
\hspace{-6mm}{\Large\bf{Literature Report}}
\par
\vspace{7pt}
Transient science with the Large Synoptic Survey Telescope (LSST) will be highly dependent on the ability to characterize the target populations of time-varying events. It will also be highly dependent on the cadence with which the survey area is sampled. The restrictions placed on these factors include our limited knowledge about the physics of these transient systems, the competition with other science cases for specific observing strategies and the physical limitations imposed by observing conditions and instrumentation. Before addressing the physical drivers for determination of the cadence from the perspective of transient science it is useful to review what the baseline cadence is and what some of the main science drivers are behind the project. \par
\section{General Cadence Considerations} %1.5 pages
The observational strategy, or cadence, that will be adopted for the LSST survey is still a work in progress. However, there are some main features which will be consistent throughout the optimization of cadence strategies. These features are primarily part of the Wide Fast Deep (WFD) survey strategy. What this actually entails are the generic features of the cadence which make this wide, fast, and deep. Naively these characteristics would seem to be something that any telescope would want the capability of having, but in this context it signifies that we are now in an era of Astronomy where all these goals can be achieved simultaneously. Covering a large sky area at a pace that would allow for the detection of changes in the sky on short timescales out to a distance that is beyond our local galactic environment make this a WFD survey \citep{LSSTScienceCollaboration2009}. \par
One of the main features of LSST that allows this to be accomplished is the telescope's large etendue, a measure of field of view (FOV) and light gathering power. LSST will have a larger etendue by order of magnitude than any previous survey. If etendue is a fiducial metric for science potential then the LSST will be a prolific generator of science. For reference, the FOV of the LSST will be 9.6 \textit{deg}$^2$. This large size a was requirement so that image quality would be atmospheric seeing limited, rather than optics limited \citep{LSSTScienceCollaboration2009}.    \par
The baseline cadence is not only the WFD survey, but it has been decided to set aside some of the available observing for mini-surveys such as the Deep Drilling Fields (DDF). While this is not the only mini survey it is one with the highest priority of the additional surveys that will be completed. This is also the only mini survey, out of those proposed in the current version of the Observing White Paper, which will help meet this science goals of the possible projects with which I will become involved. The DDF refer to a small selection of single telescope points spread throughout the sky that will be surveyed a much higher pace than the normal fields of the WFD survey. This is primarily to sample the are more frequently for early detection and better light curve sampling of transients. Additionally, though not the primary intention, this enhanced sampling will provide a much higher magnitude coadded depth probing the earlier universe. \par

\section{Physical Constraints} % 3 page


\subsection{Supernovae}

\subsection{Kilonovae}

\section{Addressing the Detection of Kilonovae with LSST} % 2 page

\subsection{Photometric Classification}

\section{Areas for Future Work} % 1 page
Better understanding of what a representative sample for Kilonovae. 
Understand the desired sampling for photometric classification using SNmachine. Which parts of the light curve are sampled. Does this affect SN or KN classification?
Lightcurves are quite featureless. Would color curves help? Colors do evolve faster than individual lightcurves, but show what seem to be more "features". 


\bibliography{PhD-Main_Project}
\bibliographystyle{aasjournal}

\end{document}

